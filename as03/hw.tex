% Set up the document
\documentclass{article}

% Page size
\usepackage[
    letterpaper,]{geometry}
\usepackage{changepage}

% Lines between paragraphs
\setlength{\parskip}{\baselineskip}
\setlength{\parindent}{0pt}

% Math
\usepackage{mathtools}
\usepackage{amssymb}
\usepackage{amsthm}
\usepackage{commath}

% Shortcut for boldface
\def\*#1{\mathbf{#1}}

% Operators
%\DeclareMathOperator{\rank}{rank}

% Asymptotic notations
\newcommand{\Oh}{\mathcal{O}}

% Number sets
\newcommand{\C}{\mathbb{C}}
\newcommand{\N}{\mathbb{N}}
\newcommand{\Q}{\mathbb{Q}}
\newcommand{\R}{\mathbb{R}}
\newcommand{\Z}{\mathbb{Z}}

% Links
\usepackage{hyperref}

% Graphics
\usepackage{float}
\usepackage{graphicx}
\graphicspath{ {./img/} }

% Page numbers at top right
\usepackage{fancyhdr}
\pagestyle{fancy}
\fancyhf{}
\fancyhead[R]{\thepage}
\renewcommand\headrulewidth{0pt}

\begin{document}

\textbf{AMATH 740 assignment 3} \\
\textbf{Matt Wiens} \\
\textbf{2020-11-27}

1. \textbf{CG step length.}
Consider the CG method for $A \*x = \*b$ with $A$ SPD. Show that, in the update formula
%
\begin{equation*}
    \*x_k = \*x_{k - 1} + \alpha_k \*p_{k - 1}
    ,
\end{equation*}
%
CG chooses the step length that minimizes $\phi(\*x)$ along the along direction $\*p_{k - 1}$,
as in steepest descent:
%
\begin{equation*}
    \dod{}{\alpha_k} \phi(\*x_k(\alpha_k)) = 0.
\end{equation*}

(a) Show first that this requires step length
%
\begin{equation*}
    \alpha_k = \frac{\*r_{k - 1}^T \*p_{k - 1}}{\*p_{k - 1}^T A \*p_{k - 1}}
    .
\end{equation*}
%
\textit{Solution.}

\vspace{5mm}

(b) Using properties we showed in class for the CG method, show that this $\alpha_k$
equals the $\alpha_k$ used in the CG algorithm:
%
\begin{equation*}
    \alpha_k = \frac{\*r_{k - 1}^T \*r_{k - 1}}{\*p_{k - 1}^T A \*p_{k - 1}}
    .
\end{equation*}
%
\textit{Solution.}

\newpage

2. \textbf{High-order ODE.}
Write the following third order ODE as a first-order ODE system:
%
\begin{equation*}
    y^{\prime \prime \prime}(x) + 3 y^{\prime \prime}(x) - 4 y^\prime(x) + 7 y(x) = x^2 + 7,
\end{equation*}
%
and give the system in matrix form.

\textit{Solution.}

\newpage

3. \textbf{Local truncation error for Ralston method.}
Consider the Ralson method for ODE $y^\prime = f(x, y)$:
%
\begin{align*}
    k_1 &= f(x_n, y_n), \\
    k_2 &= f\del{x_n + \frac{2}{3} h, y_n + \frac{2}{3}h k_1}, \\
    y_{n + 1} &= y_n + h \del{\frac{1}{4} k_1 + \frac{3}{4} k_2}.
\end{align*}
%
Show that the local truncation error at $x_{n + 1}$, given by
$l_{n + 1} = \hat{y}(x_{n + 1}) - y_{n - 1}$, is $\Oh(h^3)$.
(Note: We assume as usual that $f$ and its derivatives are sufficiently smooth and bounded. The Ralson method is a 2-stage RK method.)

\textit{Solution.}

\newpage

4. \textbf{Local truncation error for $\boldsymbol\eta$ method}.
Consider numerical method
%
\begin{equation*}
    y_{n + 1} = y_n + h f(x_n + (1 - \eta) h, \eta y_n + (1 - \eta) y_{n + 1}), \quad \eta \in [0, 1]
\end{equation*}
%
for $y^\prime = f(x, y)$. Show that the local truncation error $l_{n + 1} = \Oh(h^2)$ for any $\eta \in [0, 1]$. Is there
a value of $\eta$ for which $l_{n + 1} = \Oh(h^3)$?

(Note: We assume as usual that $f$ and its derivatives are sufficiently smooth and bounded.)

\textit{Solution.}

\newpage

5. \textbf{Accurate approximation at $\boldsymbol{x_1}$}.
You are given the ODE
%
\begin{equation*}
    y^\prime(x) = f(x, y)
\end{equation*}
%
with explicit knowledge of $f(x, y)$ as a function of $x$ and $y$. Assume that, given the initial condition $y_0 = y(x_0)$,
you need a starting value for $y_1$ at $x_1 = x_0 + h$ that is accurate with order $h^4$. Find an explicit method for calculating
such an accurate starting value $y_1$, using only evaluations of $f(x, y)$ and its partial derivatives at $x_0$.

\textit{Solution.}
(hint: use Taylor series)

\newpage

6. \textbf{Numerical stability of backward Euler method}.
Find the region $D$ of absolute stability for the backward Euler method
%
\begin{equation*}
    y_{n + 1} = y_n + h f(x_{n + 1}, y_{n + 1})
\end{equation*}
%
in the complex $h \lambda$ plane.

(Note: this is an implicit one-step method, and can be derived similar to the forward Euler method.)

\textit{Solution.}

\newpage

7. \textbf{Numerical stability of Ralston method.}
Consider Ralson's method
%
\begin{align*}
    k_1 &= f(x_n, y_n), \\
    k_2 &= f\del{x_n + \frac{2}{3} h, y_n + \frac{2}{3}h k_1}, \\
    y_{n + 1} &= y_n + h \del{\frac{1}{4} k_1 + \frac{3}{4} k_2}.
\end{align*}
%
(a) Show that the region $D$ of absolute stability for this method in the complex $h \lambda$ plane is
described by the condition
%
\begin{equation*}
    |1 + h \lambda + h^2 \lambda^2 / 2| < 1.
\end{equation*}
%
\textit{Solution.}

\vspace{5mm}

(b) Consider this inequality for the case of real $\lambda < 0$, and derive an upper stability bound for the step length $h$.

\textit{Solution.}

\end{document}
