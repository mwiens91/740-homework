% Set up the document
\documentclass{article}

% Page size
\usepackage[
    letterpaper,]{geometry}
\usepackage{changepage}

% Lines between paragraphs
\setlength{\parskip}{\baselineskip}
\setlength{\parindent}{0pt}

% Math
\usepackage{mathtools}
\usepackage{amssymb}
\usepackage{amsthm}
\usepackage{commath}

% Shortcut for boldface
\def\*#1{\mathbf{#1}}

% Number sets
\newcommand{\C}{\mathbb{C}}
\newcommand{\N}{\mathbb{N}}
\newcommand{\Q}{\mathbb{Q}}
\newcommand{\R}{\mathbb{R}}
\newcommand{\Z}{\mathbb{Z}}

% Links
\usepackage{hyperref}

% Page numbers at top right
\usepackage{fancyhdr}
\pagestyle{fancy}
\fancyhf{}
\fancyhead[R]{\thepage}
\renewcommand\headrulewidth{0pt}

\begin{document}

\textbf{AMATH 740 assignment 1} \\
\textbf{Matt Wiens} \\
\textbf{2020-08-25}

1. \textbf{Eigenvalues and eigenvectors of the 1D Laplacian.}
Consider the finite difference matrix operator $A \in \R^{n \times n}$ for the 1D model problem
%
\begin{equation*}
    \dod[2]{u(x)}{x} = f(x)
\end{equation*}
%
on domain $[0, 1]$ with boundary conditions $u(0) = 0$ and $u(1) = 0$,
given by
%
\begin{equation*}
    A = \frac{1}{h^2}
        \begin{bmatrix}
            -2 & 1 & & & & & \\
            1 & -2 & 1 & & & \\
              & 1 & -2 & 1 & & \\
              & & \ddots & \ddots & \ddots & \\
              & & & 1 & -2 & 1 \\
              & & & & 1 & -2
         \end{bmatrix}
         ,
\end{equation*}
%
where
%
\begin{equation*}
    h = \frac{1}{n + 1}
    .
\end{equation*}
%
This matrix can be considered a discrete version of the continuous operator
$\od[2]{}{x}$ that acts on a function $u(x)$.

(a) Show that the $n$ eigenvectors of $A$ are given by the vectors
$\*x^{(p)}$ (with $p = 1, \ldots, n$) with components
%
\begin{equation*}
    x_i^{(p)} = \sin(p \pi i h),
\end{equation*}
%
and with eigenvalues
%
\begin{equation*}
    \lambda_p = \frac{2}{h^2} \del{\cos(p \pi h) - 1}
    .
\end{equation*}

\textit{Solution.}
hint is to verify general components $A \*x^{(p)}$ using trig idents

\vspace{5mm}

(b) Verify that the functions $u^{(p)}(x) = \sin(p \pi x)$ (with
$p \in \N$) are eigenfunctions of the continuous differential operator
$\od[2]{}{x}$ on domain $[0, 1]$ with boundary conditions
$u(0) = 0$ and $u(1) = 0$. What are the eigenvalues?

\textit{Solution.}
There's bit about what an eigenvalue must satisfy in teh hw handout

\vspace{5mm}

(c) Compare the eigenvectors and eigenvalues for the discrete and the
continuous operators and comment. Are the discrete and continuous eigenvalues
similar for small values of $hp$?

\textit{Solution.}
there's a bit about what they want you to comment about in the hw handout

\newpage

2. \textbf{LU decomposition.}
Find the LU decomposition of
%
\begin{equation*}
    A = 
    \begin{bmatrix}
        1 & 4 & 7 \\
        2 & 5 & 8 \\
        3 & 6 & 10
    \end{bmatrix}
    .
\end{equation*}
%
Briefly explain the steps.

\textit{Solution.}
hey

\newpage

3. \textbf{Computational work for recursive determinant computation.}
Determine the number of flops required for calculating the determinant
of an $n \times n$ matrix using a straightforward implementation of the recursive
definition of the determinant. You can combine additions and multiplications.
Find an approximation for the expression derived that is valid for large $n$.

\textit{Solution.}
there's some hints about how to do this, and what you might use for the approximate expression

\newpage

4. \textbf{Vector norm inequalities.}
Let $\*x \in \R^n$. Show that
$\enVert{\*x}_\infty \leq \enVert{\*x}_1 \leq n \enVert{\*x}_\infty$.

\textit{Solution.}
hey

\newpage

5. \textbf{Matrix norm formula.}
Let $A \in \R^{n \times n}$. Show that
%
\begin{equation*}
    \enVert{A}_1 = \max_{1 \leq j \leq n} \sum_{i = 1}^n |a_{ij}|
\end{equation*}
%
(this is the \textit{maximum absolute column sum}).

\textit{Solution.}
hey

\newpage

6. \textbf{Inverse update formula.}
Let $A \in \R^{n \times n}$ be a nonsingular matrix, and $\*u, \*v \in \R^n$.
Show that if $A + \*u \*v^T$ is nonsingular, then its inverse can be expressed
by the formula
%
\begin{equation*}
    (A + \*u \*v^T)^{-1} = A^{-1} - \frac{1}{1 + \*v^T A^{-1} \*u} A^{-1} \*u \*v^T A^{-1}.
\end{equation*}
%
Note: you also have to show that $1 + \*v^T A^{-1} \*u \neq 0$, because otherwise the formula
would be ill-defined.

\textit{Solution.}
there's a hint in the homework handout about how to do the note (basically prove by contradiction)

\end{document}
